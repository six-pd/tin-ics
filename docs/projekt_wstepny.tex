\documentclass{article}
\usepackage{polski}
\usepackage[utf8]{inputenc}
\usepackage{url}

\author{Bartłomiej Partyka - lider, Michał Urbański, Krzysztof Blankiewicz, Tomasz Załuska}

\title{Projekt wstępny TIN}

\begin{document}

\maketitle

\section{Spis treści\label{spis}}

\begin{itemize}

    \item \ref{spis}. Spis treści . . . . . . . . . . . . . . . . . . . . . . . . . . . . . . . . . . . . . . . . . . \pageref{spis}
    \item \ref{tresc}. Treść zadania . . . . . . . . . . . . . . . . . . . . . . . . . . . . . . . . . . . . . . . . \pageref{tresc}
    \item \ref{przypf}. Założenia funkcjonalne i niefunkcjolanle . . . . . . . . . . . . . . . . . . . . . \pageref{przypf}
    \item \ref{przyp}. Przypadki użycia . . . . . . . . . . . . . . . . . . . . . . . . . . . . . . . . . . . . . \pageref{przyp}
    \item \ref{srod}. Środowisko . . . . . . . . . . . . . . . . . . . . . . . . . . . . . . . . . . . . . . . . . \pageref{srod}
    \item \ref{arch}. Architektura rozwiązania . . . . . . . . . . . . . . . . . . . . . . . . . . . . . . . \pageref{arch}
    \item \ref{test}. Sposób testowania . . . . . . . . . . . . . . . . . . . . . . . . . . . . . . . . . . . . \pageref{test}
    \item \ref{prez}. Prezentacja . . . . . . . . . . . . . . . . . . . . . . . . . . . . . . . . . . . . . . . . . \pageref{prez}
    \item \ref{prac}. Podział pracy . . . . . . . . . . . . . . . . . . . . . . . . . . . . . . . . . . . . . . . \pageref{prac}
    \item \ref{harm}. Harmonogram pracy . . . . . . . . . . . . . . . . . . . . . . . . . . . . . . . . . . \pageref{harm}
    \item \ref{git}. Repozytorium github . . . . . . . . . . . . . . . . . . . . . . . . . . . . . . . . . \pageref{git}


\end{itemize}

\section{Treść zadania\label{tresc}}

Urządzenia przechowują zmienne statyczne w pamięciach ulotnej i nieulotnej. Zmienne dynamiczne przechowywane są w pamięci ulotnej. Projektowany system komunikacji działa niezależnie od systemu wykrywania zaniku zasilania i zachowywania stanu urządzenia. Aplikacja korzystająca z usług tego systemu potwierdza konsumpcję danych. Celem systemu jest retransmisja niepotwierdzonych danych po wznowieniu zasilania. System używa stosu TCP/IPv6. Zaprojektować API systemu komunikacyjnego. Ponadto, należy zaprojektować moduł do Wireshark umożliwiający wyświetlanie i analizę zdefiniowanych komunikatów. (Być może pomocnym będzie przejrzenie RFC 5326 "Licklider Transmission Protocol").

\section{Nazwa własna\label{nazwa}}

ICS - Indomitable Communications System
\section{Założenia funkcjonalne i niefunkcjonalne\label{przypf}}
\subsection{Założenia funkcjonalne}

System ICS używa IPv6 jako protokołu warstwy trzeciej.


Klient:
\begin{itemize}
\item Łączy się z serwerem komunikacji
\item Wyświetla listę dostępnych klientów
\item Akceptuje połączenie z innym klientem lub sam inicjuje połączenie
\item Ma możliwość wysyłania i odbierania dużych plików oraz komunikatów tekstowych
\item W wypadku zaniku zasilania/połączenia jest w stanie bez przeszkód wznowić komunikację
\item Ma możliwość utrzymywania wielu połączeń jednocześnie
\end{itemize}
Serwer:
\begin{itemize}
\item Umożliwia nawiązywanie połączeń między klientami
\item Umożliwia autoryzację użytkowników poprzez hasło serwerowe
\item Przechowuje w pamięci ulotnej informację o przesłanych danych aby umożliwić wznowienie komunikacji
\item Komunikacja jest utrzymywana przez zaniki połączenia aż do zamknięcia połączenia przez użytkownika lub po określonym w konfiguracji czasie nieaktywności
\item Kontrola przesyłania danych realizowana jest przez specjalnie zaprojektowane API
\end{itemize}
Moduł wireshark:
\begin{itemize}
\item Służy do testów i debugowania aplikacji
\item Wspomaga proces tworzenia systemu komunikacji
\end{itemize}

\subsection{Założenia niefunkcjonalne}

\begin{itemize}
\item Niezawodna komunikacja
\item Prosty w obsłudze interfejs klienta w linii komend (TUI/CLI)
\item Możliwość konfiguracji parametrów serwera
\item Możliwość ustawiania parametrów połączenia przez klienta
\end{itemize}


\section{Przypadki użycia\label{przyp}}

\begin{enumerate}
\item Klient łączy się z serwerem, loguje i wybiera z listy innego klienta, z którym inicjuje połączenie.
Każdy klient może być połączony z kilkoma klientami jednocześnie.
\item Klient może wysłać wiadomość (tekst lub plik) do dowolnego klienta, z którym jest połączony.
\item Wiadomość po podzieleniu na bloki jest wysyłana na serwer, który podczas odbierania danych przechowuje informację o ilości odebranych dotychczas danych. W przypadku zaniku i wznowieniu zasilania, po ponownym połączeniu się z serwerem nadawca automatycznie wznawia wysyłanie wiadomości.
\item Po udanym wysłaniu pliku przez nadawcę serwer rozpoczyna przesyłanie pliku odbiorcy, prosząc użytkownika o potwierdzenie odbioru każdego wysłanego bloku. Serwer przechowuje informację o ilości poprawnie wysłanych danych, w przypadku przerwania, po ponownym połączeniu, serwer automatycznie wznawia wysyłanie do klienta.
\item Po udanej transmisji wiadomość jest usuwana z serwera (lub po pewnym określonym czasie)
\item Transmisja wiadomości może być anulowana w każdej chwili przez obie strony. 
\item Odebrany plik zapisuje się na dysku odbiorcy, odebrany tekst wyświetla się w konsoli.
\end{enumerate}

Użytkownik może się w dowolnej chwili rozłączyć z serwerem.

\section{Środowisko\label{srod}}

\subsection{Systemy operacyjne}

Projekt tworzony jest dla systemów Linux. Użyte dystrybucje to:
\begin{itemize}

\item Void Linux 64-bit, kernel 5.4
\item Debian Linux 10 64-bit, kernel 4.19
\item Arch Linux 64-bit, kernel 5.5.10
\item Ubuntu Linux

\end{itemize}

\subsection{Języki i biblioteki}

Klient i serwer będą pisane w języku C/C++, z wyjątkiem konsolowego interfejsu użytkownika który będzie pisany w języku Python.

Używane biblioteki:
\begin{itemize}

\item boost
\item pthread
\item ncurses
  
\end{itemize}


Moduły do Wireshark napisane w języku Lua.

\subsection{Narzędzia testowe}

Jedym z ważniejszych narzędzi testowych będzie Wireshark, wraz z specjalnie napisanym modułem do analizy komunikacji w systemie.
 Do testowania funkcji programu posłuży biblioteka boost.

\section{Architektura rozwiązania\label{arch}}

System ICS składa się z następujących modułów:
\begin{itemize}
\item Serwera
\item Aplikacji klienckiej
\item Interfejsu użytkownika
\item Modułu Wireshark
\end{itemize}

\textbf{Serwer} ICS ma za zadanie zawiązywać, kontrolować i utrzymywać połączenia między klientami. Umożliwia on również podstawową autoryzację w postaci hasła. Podczas transferu danych, serwer i klient potwierdzają odbiór bloków danych i są w stanie wznowić transmisję po awarii lub utracie połączenia. Serwer ics posiada plik konfiguracyjny, aby umożliwić administratorowi dostosowanie go do swoich potrzeb.


\textbf{Klient} ICS jest w stanie połączyć się z wybranym serwerem ICS, a następnie nawiązać połączenie z dowolnym klientem dostępnym na serwerze. Połączenie te umożliwia niezawodną transmisję danych -  plików lub wiadomości tekstowych. Każdy klient ma możliwość utrzymywania wielu połączeń na raz. Użytkownik wchodzi w interakcje z klientem poprzez \textbf{prosty interfejs} TUI, umożliwiający przeglądanie otwartch połączeń, przesyłanie danych i modyfikacje parametrów połączenia.


\textbf{Moduł Wireshark} tworzony jest aby ułatwić zespołowi pracę nad programem i projektowanie API do komunikacji między serwerem a klientami. Posłuży również do zaprezentowania ostatecznej wersji programu.


\section{Sposób testowania\label{test}}

Program będzie testowany pod kątem zgodności z głównym przyjętym założeniem - wznowieniem komunikacji w przypadku braków w zasilaniu, połączeniu. Zostanie do tego celu użyta specjalna, ,,developerska" wersja klienta, mająca możliwość wysyłania wybranych komunikatów i symulacji awarii poprzez interfejs użytkownika.


Podczas testowania projektu API komunikacyjnego użyty będzie program Wireshark, aby mieć dokładniejszy wgląd w dane przesyłane między serwerem a klientami.

Prowadzone będą również testy jednostkowe i regresyjne, używając biblioteki boost.


\section{Prezentacja\label{prez}}

\begin{enumerate}
    \item Pomyślna kompilacja
    \item Uruchomienie serwera i podłączenie klientów
    \item Prezentacja komunikacji między klientami
    \item Prezentacja komunikacji dla różnych przypadków utraty połączenia/zasilania.
    \item Obserwacja ruchu programem Wireshark
\end{enumerate}
    

\section{Podział pracy\label{prac}}
\begin{itemize}
\item Bartłomiej Partyka - lider - Moduł Wireshark, Interfejs, Klient.

\item Michał Urbański - Testy, boost, makefile, pomoc w pracy nad głównymi modułami.

\item Krzysztof Blankiewicz - API komunikacji, Serwer(niezawodna komunikacja).

\item Tomasz Załuska - Gniazda, Serwer(zarządzanie połączeniami). 
\end{itemize}

\section{Harmonogram pracy\label{harm}}

Proponowany jest następujący harmonogram pracy:


25.04.2020  - Prototyp - działająca aplikacja klienta i serwera umożliwiające komunikację


10.05.2020  - Działająca wersja wersja z zaimplementowanym mechanizmem przeciwdziałania zanikom zasilania

\section{Repozytorium github\label{git}}

Projekt jest publicznie dostępny pod adresem: \url{https://github.com/six-pd/tin-ics}

\end{document}
